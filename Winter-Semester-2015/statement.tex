%%%%%%%%%%%%%%%%%%%%%%%%%%%%%%%%%%%%%%%%%
% Short Sectioned Assignment
% LaTeX Template
% Version
 1.0 (5/5/12)
%
% This template has been downloaded from:
% http://www.LaTeXTemplates.com
%
% Original author:
% Frits Wenneker (http://www.howtotex.com)
%
% License:
% CC BY-NC-SA 3.0 (http://creativecommons.org/licenses/by-nc-sa/3.0/)
%
%%%%%%%%%%%%%%%%%%%%%%%%%%%%%%%%%%%%%%%%%

%----------------------------------------------------------------------------------------
%	PACKAGES AND OTHER DOCUMENT CONFIGURATIONS
%----------------------------------------------------------------------------------------

\documentclass[paper=a4, fontsize=11pt]{scrartcl} % A4 paper and 11pt font size

\usepackage[T1]{fontenc} % Use 8-bit encoding that has 256 glyphs
\usepackage{fourier} % Use the Adobe Utopia font for the document - comment this line to return to the LaTeX default
\usepackage[english]{babel} % English language/hyphenation
\usepackage{amsmath,amsfonts,amsthm} % Math packages

\usepackage{lipsum} % Used for inserting dummy 'Lorem ipsum' text into the template

\usepackage{sectsty} % Allows customizing section commands
\allsectionsfont{\centering \normalfont\scshape} % Make all sections centered, the default font and small caps

\usepackage{fancyhdr} % Custom headers and footers
\pagestyle{fancyplain} % Makes all pages in the document conform to the custom headers and footers
\fancyhead{} % No page header - if you want one, create it in the same way as the footers below
\fancyfoot[L]{} % Empty left footer
\fancyfoot[C]{} % Empty center footer
\fancyfoot[R]{\thepage} % Page numbering for right footer
\renewcommand{\headrulewidth}{0pt} % Remove header underlines
\renewcommand{\footrulewidth}{0pt} % Remove footer underlines
\setlength{\headheight}{13.6pt} % Customize the height of the header

\numberwithin{equation}{section} % Number equations within sections (i.e. 1.1, 1.2, 2.1, 2.2 instead of 1, 2, 3, 4)
\numberwithin{figure}{section} % Number figures within sections (i.e. 1.1, 1.2, 2.1, 2.2 instead of 1, 2, 3, 4)
\numberwithin{table}{section} % Number tables within sections (i.e. 1.1, 1.2, 2.1, 2.2 instead of 1, 2, 3, 4)

\setlength\parindent{0pt} % Removes all indentation from paragraphs - comment this line for an assignment with lots of text

%----------------------------------------------------------------------------------------
%	TITLE SECTION
%----------------------------------------------------------------------------------------
\begin{document}

\newcommand{\horrule}[1]{\rule{\linewidth}{#1}} % Create horizontal rule command with 1 argument of height

\title{	
\normalfont \normalsize 
\textsc{Universit\"{a}t Hamburg, Department of Informatics} \\ [25pt] % Your university, school and/or department name(s)
\horrule{0.5pt} \\[0.4cm] % Thin top horizontal rule
\huge Arabic Language Understanding For Robots Using Echo State Networks \\ % The assignment title
\horrule{2pt} \\[0.5cm] % Thick bottom horizontal rule
}

\author{Ali Saleh} % Your name

\date{\normalsize\today} % Today's date or a custom date


\maketitle % Print the title

%----------------------------------------------------------------------------------------
%	PROBLEM 1
%----------------------------------------------------------------------------------------

\section{Project Description}
In their paper \cite{Hinaut2013} Hinaut and Dominey demonstrated the usage of Echostate network for the task of English language understanding. After training, the network was able to understand sentences used for testing. This method were then applied to the iCub and Nao robots that made the robots able to understand actions said to it by a speaker. This method has been used for both English and French languages so far. The aim of this project to extend this to Arabic language. Explore how training the network with Arabic sentences will differ from the English and French. Finally the trained network will be tested on a robot to demonstrate the network abilities.

\section{Proposed Time Line}
The proposed time table below shows the different tasks and the time frame assigned to them in hours.
\begin{itemize}
\item 6 : Search for corpus in Arabic
\item 6 : Read papers about the sentence comprehension neural model
\item 25: Create a simple context-free grammar in Arabic that generate sentences and predicates in the robot format.
\item 10: Analyse, test, and use code (provided by Dr.Hinaut) to extract sentences and thematic roles from the generated sentences
\item 10: train the neural network with the code and explore hyperparameters with hyperopt (provided by Dr.Hinaut)
\item 108: Conduct an experiment with 5-10 subject on Arabic corpus like in the experiment \cite{Hinaut2014} to be used for the robot training.
\item 15-30: Writing a thesis, documentation source code, and a final presentation.
\end{itemize}


\section{Learning Outcomes}
\begin{itemize}
\item Learning the full research cycle in a mini-scale project.
\item Applying theoretical Neural Networks knowledge on a practical project.
\item Learn about language understanding techniques.
\end{itemize}

\bibliographystyle{plain}

\bibliography{sample}

\end{document}

