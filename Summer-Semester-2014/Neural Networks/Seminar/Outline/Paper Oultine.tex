\title{Using Recursive Neural Networks to Improve Natural Language Parsing}
\author{
        Ali Saleh \\
        Department of Computer Science\\
        Hamburg University\\            
}
\date{\today}

\documentclass[12pt]{article}

\begin{document}
\maketitle
\newpage 

\begin{abstract}
The proposed paper aims to give a brief introduction about the current advances in the field of 
Natural Language Parsing. Compare the state of the art parsing technique with a parser that  uses Recursive Neural Network on top of a Probabilistic Context-Free Grammars parser to achieve better performance.
\end{abstract}

\section{Introduction}
Natural language Parsing is a preliminary step to any serious NLP research to be able to understand the meaning of sentences you have to first divide them into sentence part. The approach to be examined by the proposed paper is trying to combine the abilities of the PCFGs to categorize the sentence parts (NP, PP , VP,\ldots) and the abilities of the RNNs to capture the syntactic syntactic and compositional-semantic information from the sentence. This combination is said to improve the overall performance of the parsing process by adding more information improve the categorization of the sentence parts.

\paragraph{Paper Outline}
The proposed paper will be structured as following:
\begin{itemize}
\item Introduction
\item Detailed explanation of the Compositional Vector Grammars approach
\item Brief summary of other state-of-art parsing technique
\item Performance comparison
\item Discussion and Remarks
\item Conclusion
\item References
\end{itemize}

\end{document}

  